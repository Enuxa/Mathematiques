\documentclass[]{article}
\usepackage[utf8]{inputenc}
\usepackage{pdfpages}
\usepackage{amsmath}
\usepackage{amssymb}
\usepackage{graphicx}
\usepackage{geometry}
\usepackage{enumitem}
\usepackage{amsthm}
\usepackage{stmaryrd}

\geometry{hmargin=2cm}

\title{Analyse II - Probabilités et intégrale de Lebesgue - Notes prises par Pierre Gervais}
\author{Marc Rosso}

% Environnement type théorème
\newtheorem{mythm}{Théorème}
\newtheorem{myproposition}{Proposition}
\newtheorem{myproperty}{Propriété}
\newtheorem{mylemma}{Lemme}

% Environnement type texte
\theoremstyle{remark}
\newtheorem{mynot}{Notation}
\newtheorem{myrem}{Remarque}
\newtheorem{myexer}{Exercice}
\newtheorem{myproof}{Preuve}
\newtheorem{myexmpl}{Exemple}

% Environnement de définition
\theoremstyle{definition}
\newtheorem{mydef}{Définition}
\newtheorem{myquestion}{Question}

\setlist[itemize]{label=-}

% Carré de fin de preuve
\newcommand{\cqfd}{
	\hfill$\square$
}

% Définition de fonction
\newcommand{\func}[5]{
#1 ~ : ~ \left\{ \begin{array}{lcl}
	#2 & \longrightarrow & #3 \\
	#4 & \longmapsto & #5
\end{array}
\right.
}

\newcommand{\fun}[3]{
#1 ~ : ~ #2 \longrightarrow #3
}

\newcommand{\funcinline}[5]{
	#1 \, : \, #2 \longrightarrow #3, ~ #4 \longmapsto #5
}

\newcommand{\funcshort}[3]{
	#1 \, : \, #2 \longrightarrow #3
}

\newcommand{\anonfunc}[4]{
	\left\{ \begin{array}{lcl}
		#1 & \longrightarrow & #2 \\
		#3 & \longmapsto & #4
	\end{array}
	\right.
}

\newcommand{\prob}[1]{\mathbb{P}\left(#1\right)}

\newcommand{\card}{\text{Card }}

\newcommand{\DS}{\displaystyle}

\begin{document}

\maketitle

\tableofcontents

\newpage

\section{Cadre mathématiques pour l'aléatoire}

\begin{myquestion}
	En jetant une aiguille de taille $2a$ sur un parquet dont les lattes sont de largeurs $2 \ell$, quelle est la probabilité que l'aiguille tombe sur une rainure ? (avec $a \leqslant \ell$).
	
	On note $r$ la distance du milieu de l'aiguille à la rainure la plus proche ($0 \leqslant r \leqslant \ell$), et $\alpha$ l'angle des droites entre l'aiguille et la direction de la rainure ($0 \leqslant \alpha \leqslant \pi/2$).	
	
	Pour $r$ donné, et $\alpha_0$ la valeur de contact on a $$r = a \sin \alpha_0$$
	
	La condition sur la position $(r, \alpha)$ de l'aiguille pour couper la rainure est $$r \leqslant a \sin \alpha$$
	
	Soit $D := \{(r, \alpha) \in [0, \ell] \times [0, \pi / 2] ~ | ~ r \leqslant a \sin \alpha\}$, intuitivement la probabilité associée à cet événement est le quotient $$\mathbb{P}(D) = \frac{Aire(D)}{l \cdot \pi / 2}$$ où $$Aire(D) = \int_{0}^{\pi/2} \int_{0}^{a sin \alpha}dr d\alpha = a [-\cos \alpha]_0^{\pi/2} = a$$ enfin $\mathbb{P}(D) = \frac{a}{l} \cdot \frac{2}{\pi}$
\end{myquestion}

\section{Algèbre de Boole, tribus et mesures de probabilités}

\begin{mydef}
	Soit $\Omega$ un ensemble, une \textit{algèbre de Boole} $\mathcal{A} \subseteq \mathcal{P}(\Omega)$ est une classe de parties de $\Omega$ contenant $\Omega$ et stable par passage au complémentaire et par réunion.
\end{mydef}

\newpage

\begin{myexmpl}
	\leavevmode
	\begin{enumerate}
		\item $\mathcal{A} = \mathcal{P}(\Omega)$
		\item $\mathcal{A} = \{\Omega, \emptyset\}$
		\item Si $A \subseteq \Omega$, $\mathcal{A} = \{A, A^c, \emptyset, \Omega\}$
	\end{enumerate}
\end{myexmpl}

\begin{myproposition}
	Soit $\mathcal{A}$ une algèbre de Boole
	\begin{enumerate}
		\item $\emptyset \in \mathcal{A}$
		\item $\mathcal{A}$ est clos par intersection
		\item $\mathcal{A}$ est clos par intersection et réunion finie
	\end{enumerate}
\end{myproposition}

\begin{mydef}
	Soit $\Omega$ un ensemble, une \textit{tribu} ou \textit{$\sigma$-algèbre}$\mathcal{A}$ est une classe de parties de $\Omega$ telle que
	\begin{enumerate}
		\item $\Omega \in \mathcal{A}$ (on l'appelle \textit{l'univers})
		\item $\mathcal{A}$ soit stable par passage au complémentaire
		\item $\mathcal{A}$ soit stable par union dénombrable
	\end{enumerate}
\end{mydef}

\begin{myproposition}
	Soit $\mathcal{A}$ une $\sigma$-algèbre
	\begin{enumerate}
		\item $\emptyset \in \mathcal{A}$
		\item $\mathcal{A}$ est stable par intersection
		\item $\mathcal{A}$ est stable par intersection dénombrable
	\end{enumerate}
\end{myproposition}

\begin{myexmpl}
	\leavevmode
	\begin{itemize}
		\item $\mathcal{P}(\Omega)$
		\item $\{\emptyset, \Omega\}$
	\end{itemize}
\end{myexmpl}

\begin{myrem}
	\leavevmode
	\begin{enumerate}
		\item Soit $(A_n)_{n \in \mathbb{N}}$ une suite de parties de $\Omega$ et $E = \{\omega \in \Omega ~ | ~ \omega \text{ est dans une infinité d'ensembles $A_n$} \}$, on a
		$$E = \bigcap_{n \in \mathbb{N}} \left(\bigcup_{k \geqslant n} A_k\right) = \limsup A_n$$
		
		La suite $\displaystyle B_n = \bigcup_{k \geqslant n} A_k = A_n \cup B_{n+1}$ est décroissante.
		
		\item $F = \{\omega \in \Omega ~ | ~ \text{$\omega$ est dans tous les $A_n$ sauf un nombre fini d'entre eux} \}$
		
		$\displaystyle \omega \in \bigcup_{n \in \mathbb{N}} \left(\bigcap_{k \geqslant n} A_k \right) = \liminf (A_n)$
	\end{enumerate}
\end{myrem}

\begin{mydef}
	Soit $\Omega$ un ensemble, $\mathcal{C} \subseteq \mathcal{P}(\Omega)$ une classe de parties de $\Omega$. Il existe une plus petite tribu contenant $\mathcal{C}$, appelée \textit{tribu engendrée par $\mathcal{C}$} et notée $\sigma(\mathcal{C})$, en notant $\mathcal{T}$ l'ensemble des tribus contenant $\mathcal{C}$ :
	
	$$\sigma(\mathcal{C}) = \bigcap_{\mathcal{A} \in \mathcal{T}} \mathcal{A}$$
\end{mydef}

\begin{myrem}
	\leavevmode
	\begin{itemize}
		\item Il y a toujours au moins une tribu qui contient $\mathcal{C}$ : $\mathcal{P}(\Omega)$
		\item Si $\mathcal{A}_1$ et $\mathcal{A}_2$ sont deux tribus, $\mathcal{A}_1 \cap \mathcal{A}_2$ est encore une tribu.
	\end{itemize}
\end{myrem}

\begin{myexmpl}
	La \textit{tribu borélienne de $\mathbb{R}$} notée $B(\mathbb{R})$ est la tribu de $\mathbb{R}$ engendrée par les intervalles fermés bornés $[a,b]$ avec $-\infty < a \leqslant b < +\infty$.
\end{myexmpl}

\begin{myproperty}
	\item $B(\mathbb{R})$ contient les intervalles ouverts et semi-ouverts :
	$$[a, b[ = \bigcup_{n > 0} [a, b - 1/n]$$
\end{myproperty}

\begin{mydef}
	Un \textit{espace probabilisable} est un ensemble $\Omega$ muni d'une tribu $\mathcal{A} \subseteq \mathcal{P}(\Omega)$.
	
	Une \textit{probabilité} (ou \textit{mesure de probabilité}) sur $(\Omega, \mathcal{A})$ est une application $\funcshort{\mathbb{P}}{\mathcal{A}}{[0, 1]}$ vérifiant les propriétés suivantes :
	\begin{itemize}
		\item \textit{masse unitaire} : $\mathbb{P}(\Omega)=1$
		\item \textit{$\sigma$-additivité} : si $(A_n)_n$ est une suite d'éléments disjoints de $\mathcal{A}$, alors $$\mathbb{P}\left(\bigcup_{n \geqslant 0} A_n\right) = \sum_{n = 0}^{\infty} \mathbb{P}(A_n)$$
	\end{itemize}
\end{mydef}

\begin{myrem}
	L'ordre de sommation de la suite $(\prob{A_n})_n$ n'est pas important car on a là une série à termes positifs, donc absolument convergente et un théorème nous indique que pour toute série absolument convergente $\sum a_n$ et toute permutation $\sigma$ de $\mathbb{N}$, on a $$\sum a_n = \sum a_{\sigma(n)}$$
\end{myrem}

\begin{myproposition}
	Soit $\mathbb{P}$ une probabilité sur $(\Omega, \mathcal{A})$, alors
	\begin{enumerate}
		\item $\mathbb{P}(\emptyset) = 0$
		\item Si $A_1, A_2, ... A_n$ sont deux à deux disjoints, $\mathbb{P}(A_1 \cup A_2 \cup ... A_n) = \mathbb{P}(A_1) + \mathbb{P}(A_2) + ... \mathbb{P}(A_n)$
		\item $\mathbb{P}(A^c) = 1 - \mathbb{P}(A)$
		\item Pour tout $A, B \in \mathcal{A}$ : $\mathbb{P}(A \cup B) = \mathbb{P}(A) + \mathbb{P}(B) - \mathbb{P}(A \cap B)$
		\item Si $A \subseteq B$ alors $\mathbb{P}(A) \leqslant \mathbb{P}(B)$
		\item Si $(A_n)_n$ est une suite de parties dans $\mathcal{A}$ qui ne sont pas nécessairement disjointes
		$$\mathbb{P}\left(\bigcup_{n \geqslant 0} A_n\right) \leqslant \sum_{0}^{\infty} \mathbb{P}(A_n)$$
	\end{enumerate}
\end{myproposition}

\begin{myproof}
		\leavevmode

		Montrons le point 4 : soient $A, B \in \mathcal{A}$, on a $A \cup B = (A) \sqcup (B \setminus (A \cap B))$ d'où $\prob{A \cup B} = \prob{A} + \prob{B \setminus (A \cap B)}$, or $B = (A \cap B) \sqcup (B \setminus (A \cap B))$, alors $\prob{B} = \prob{A \cap B} + \prob{B \setminus (A \cap B)}$ et donc
		
		$$\prob{A \cup B} = \prob{A} + \prob{B} - \prob{A \cap B}$$
	
		On en déduit 5.
	
		Montrons les points 3 et 1 : $\prob{A} = \prob{A \sqcup A^c} = \prob{A} + \prob{A^c}  = 1$, donc $\prob{A^c} = 1 - \prob{A}$.
	
		Soit $A_n$ une suite dans $\mathcal{A}$, on pose $A'_1 = A_1$, $A_2' = A_2 \setminus A_1 = A_2 \cap A_1^c$, ... $A_n' = A_n \cap (A_1 \cup A_2 .. A_{n-1})^c$. Les $A_k'$ sont deux à deux disjoints et $A_k' \subseteq A_k$ et donc $$\prob{\bigcup_{n \geqslant 0} A_n} \leqslant \sum_{n = 0}^{\infty} \prob{A_n}$$
\end{myproof}

\begin{myproposition}
	Soient $A_1, A_2, ... A_n$ des éléments de $\mathcal{A}$, on a $$\prob{A_1 \cup A_2 .. A_n} = \sum_{k = 1}^{n} (-1)^{k+1} \sum_{I \in \mathcal{P}_k(\llbracket 1, n\rrbracket)} \prob{\bigcap_{i \in I} A_i}$$
\end{myproposition}

\begin{mydef}
		Soit $\omega \in \Omega$, on pose $$\mathbb{P}_\omega(A) = \left\{
		\begin{array}{ll}
			1&\text{ si } \omega \in A \\
			0 & \text{ sinon}
		\end{array}\right.
		$$
		C'est la \textit{masse de Dirac au point $\omega$}.
		
		Soit $(\omega_n)_n$ une suite d'éléments distincts de $\Omega$ et $(\alpha_n)_n$ une suite de réels à valeurs dans $[0, 1]$ tels que $\displaystyle \sum_{n = 0}^{\infty} \alpha_n = 1$, la fonction suivante est une probabilité :
		
		$$\mathbb{P} = \sum_{n = 0}^{\infty} \alpha_n \mathbb{P}_{\omega_n}$$
\end{mydef}

\begin{myproof}
	En effet, $\mathbb{P}$ vérifie les deux axiomes d'une mesure de probabilité :
		$$\prob{\Omega} = \sum_{n = 0}^{\infty} \alpha_n \underbrace{\mathbb{P}_{\omega_n} (\Omega)}_{1} = \sum_{n = 0}^{\infty} \alpha_n = 1$$
		et pour toute suite d'éléments $(A_n)_n$ de $\mathcal{A}$ deux à deux disjoints :
		$$\begin{aligned}
			\prob{\bigcup_{k \geqslant 0} A_k} &= \sum_{n = 0}^{\infty} \alpha_n \mathbb{P}_{\omega_n} \left(\bigcup_{k \geqslant 0} A_k\right) \\
			&= \sum_{n=0}^{\infty} \alpha_n \sum_{k = 0}^{\infty} \mathbb{P}_{\omega_n}(A_k) \\
			&= \sum_{n = 0}^{\infty} \sum_{k = 0}^{\infty} \alpha_n \mathbb{P}_{\omega_n}(A_k) \\
			&= \sum_{k = 0}^{\infty} \sum_{n = 0}^{\infty} \alpha_n \mathbb{P}_{\omega_n}(A_k) ~ \text{car il s'agit d'une série à termes positifs} \\
			&= \sum_{k = 0}^{\infty} \prob{A_k}
		\end{aligned}$$
\end{myproof}

\begin{myexmpl} Loi de Poisson : $\alpha_n = \frac{\lambda^n}{n!} e^{- \lambda}$
\end{myexmpl}

\begin{myexmpl} Équiprobabilité sur un ensemble fini
	
	Soient $\Omega = \{\omega_1, ... \omega_n\}$, $\mathcal{A} = \mathcal{P}(\Omega)$, $\displaystyle \mathbb{P} = \frac{1}{n} \sum_{k = 1}^{n} \mathbb{P}_{\omega_k}$ et $\alpha_k = 1/n$.
	
	On retrouve $\displaystyle \prob{A} = \frac{1}{n} \sum_{k = 1}^{n} \mathbb{P}_{\omega_k}(A) = \frac{|A|}{n}$
\end{myexmpl}

\begin{myrem}
	Lorsque $\Omega$ est un ensemble fini ou dénombrable, on prendra souvent $\mathcal{A} = \mathcal{P}(\Omega)$.
\end{myrem}

\begin{myexmpl} Le paradoxe de Bertrand
	
	Soit $\mathcal{C}$ le cercle de centre $O$ et de rayon 1. On considère les cordes $[A, B]$ sur le cercle et on se pose la question suivante : \textit{quelle est la probabilité pour qu'une corde non-diamètre prise au hasard soit de longueur au moins $\sqrt{3}$ ?}
	
	La corde est entièrement déterminée par la position de son milieu $M$ car $OAB$ est un triangle isocèle et $M$ est le pied de la hauteur perpendiculaire à $(OM)$.
	
	Soit $r = OM$, la longueur de la corde $\ell$ vérifie $$\frac{\ell}{2} = \sqrt{1 - r^2}$$
	
	On déduit de $\ell/2 \geqslant \sqrt{3}$ l'inégalité $r \leqslant 1/2$.
	
	On regarde les $M \in \overline{\mathcal{B}}_{1/2}(O)$, il y a plusieurs façons de choisir la probabilité :
	\begin{enumerate}
		\item L'aire du disque : $\frac{\pi \times 1 / 4}{\pi} = 1/4$
		\item Si on paramètre $M$ par ses coordonnées polaires dans $[0, 1] \times [0, 2 \pi[$ et l'ensemble cherché $[0, 1/2] \times [0, 2 \pi[$.
		
		Si on prend la mesure produit $$\frac{1/2 \times \pi}{2 \pi} = 1/2$$
	\end{enumerate}
\end{myexmpl}

\begin{myproperty}
	Soit $(A_n)_n$ une suite à valeurs dans une tribu $\mathcal{A}$
	\begin{itemize}
		\item Si $(A_n)_n$ est croissante, alors $\DS \prob{\bigcup A} = \lim_n \prob{A_n}$
		\item Si $(A_n)_n$ est décroissante, alors $\DS \prob{\bigcap A} = \lim_n \prob{A_n}$
	\end{itemize}
\end{myproperty}

\begin{myproposition}
	Soit $\funcshort{\mathbb{P}}{\mathcal{A}}{[0, 1]}$ vérifiant
	\begin{itemize}
		\item $\prob{\Omega} = 1$
		\item $\prob{A \sqcup B} = \prob{A} + \prob{B}$ pour tout éléments disjoints de $\mathcal{A}$
		\item Si $(A_n)_n$ est une suite croissante de $\mathcal{A}$ alors $\lim\lim\limits_{n} \prob{A_n} = \prob{\bigcup A}$
	\end{itemize}
\end{myproposition}

\paragraph{Rappels}

Dans un ensemble totalement ordonné de taille $n$, il existe $\binom{n}{k}$ suites strictement croissantes et $\binom{n + k - 1}{k}$ largement croissantes.

L'équation $x_1 + x_2 + ... x_k = n$ admet $\binom{n + k - 1}{k}$ solutions dans $\mathbb{N}^k$

\subsection{Exemples de mesures de probabilités}

\paragraph{Loi de Bernoulli} : pile ou face

$\Omega=\{0, 1\}$, $\prob(\{1\}) = p \in [0, 1]$ et $\prob{\{0\}} = 1 - p$.

\paragraph{Loi binomiale}

Soit $E$ un ensemble à $N$ éléments partitionné en deux ensembles $E_1$ et $E_2$. On considère une suite de $n$ tirages avec remise. Quelle est la probabilité pour que exactement $k$ termes viennent de $E_1$ ?

$\Omega = E^n, \text{Card } \Omega = N^n$ et $\mathcal{A} = \mathcal{P}(\Omega)$

\begin{itemize}
	\item on choisit les éléments de $E_1$ qui se réalisent : $|E_1|^k$ choix
	\item on choisit à quels tirages ils se réalisent : $\binom{n}{k}$ choix
	\item on choisit les événements de $E_2$ qui se réalisent : $|E_2|^{n - k}$ choix
\end{itemize}

en divisant par le cardinal de $\Omega$ et en notant $M = \card E_1$ on a comme probabilité pour un tel événement :

$$\binom{n}{k} \frac{M^k(N-M)^{n - k}}{N^n} = \binom{n}{k} \left(\frac{M}{N}\right)^k\left(1 - \frac{M}{N}\right)^{n-k} = \binom{n}{k}p^k(1-p)^{n-k}$$

On peut interpréter cette loi comme un loi de probabilité sur $\{0, 1, ...n\}$ ou pour tout $k$
$$\prob{\{k\}} = \binom{n}{k}p^k(1-p)^{n-k}$$

\paragraph{Loi hypergéométrique}

On considère cette fois les tirages sans-remise, dont exactement $k$ termes appartiennent à $E_1$.

$\Omega = \mathcal{P}_n(E)$

Pour un tirage dont $k$ éléments sont dans $E_1$, on choisit $k$ éléments \textbf{distincts} de $E_1$ : $\binom{N_1}{k}$ et le reste dans $E_2$ : $\binom{N - N_1}{n - k}$ la probabilité d'un tel événelent est ainsi
$$\frac{\binom{N}{k} \binom{N - N_1}{n - k}}{\binom{N}{n}}$$

\section{Probabilités conditionnelles}



\end{document}
