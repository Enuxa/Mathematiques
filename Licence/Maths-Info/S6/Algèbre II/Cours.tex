\documentclass[]{article}
\usepackage[utf8]{inputenc}
\usepackage{pdfpages}
\usepackage{amsmath}
\usepackage{amssymb}
\usepackage{graphicx}
\usepackage{geometry}
\usepackage{enumitem}
\usepackage{amsthm}
\usepackage{stmaryrd}
\usepackage{mathtools}

\geometry{hmargin=2cm}

\title{Algèbre II}
\author{? - Notes prises par Pierre Gervais}

% Environnement type théorème
\newtheorem{mythm}{Théorème}
\newtheorem{myproposition}{Proposition}
\newtheorem{myproperty}{Propriété}
\newtheorem{mylemma}{Lemme}

% Environnement type texte
\theoremstyle{remark}
\newtheorem{mynot}{Notation}
\newtheorem{myrem}{Remarque}
\newtheorem{myexer}{Exercice}
\newtheorem{myproof}{Preuve}
\newtheorem{myexmpl}{Exemple}
\newtheorem{myrapl}{Rappel}
\newtheorem{mycor}{Corollaire}

% Environnement de définition
\theoremstyle{definition}
\newtheorem{mydef}{Définition}
\newtheorem{myquestion}{Question}

\setlist[itemize]{label=-}

% Carré de fin de preuve
\newcommand{\cqfd}{
	\hfill$\square$
}

% Définition de fonction
\newcommand{\func}[5]{
#1 ~ : ~ \left\{ \begin{array}{lcl}
	#2 & \longrightarrow & #3 \\
	#4 & \longmapsto & #5
\end{array}
\right.
}

\newcommand{\fun}[3]{
#1 ~ : ~ #2 \longrightarrow #3
}

\newcommand{\funcinline}[5]{
	#1 \, : \, #2 \longrightarrow #3, ~ #4 \longmapsto #5
}

\newcommand{\funcshort}[3]{
	#1 \, : \, #2 \longrightarrow #3
}

\newcommand{\anonfunc}[4]{
	\left\{ \begin{array}{lcl}
		#1 & \longrightarrow & #2 \\
		#3 & \longmapsto & #4
	\end{array}
	\right.
}

\newcommand{\DS}{\displaystyle}

\begin{document}

\maketitle

\tableofcontents

\newpage

\begin{myexer}
	Soit $G$ un groupe abélien agissant transitivement sur un ensemble $A$, alors $G$ agit fidèlement si et seulement s'il agit librement.
	
	Supposons qu'il agisse fidèlement, montrons qu'il agit librement, c'est-à-dire que les stabilisateurs sont réduits à l'élément neutre.
	
	Soient $x, y \in A$, l'action est transitive alors il existe $g \in G$ tel que $y = g \cdot x$, leurs stabilisateurs sont alors conjugués. Or $G$ est abélien, donc leurs stabilisateurs sont égaux. De plus l'intersection des stabilisateurs est égale à tout stabilisateur, et celle-ci est réduite à l'élément neutre ar hypothèse. L'action est donc libre.
\end{myexer}

\section{Espaces affines}

\begin{mydef}
	Deux espaces affines $\mathcal{E}$ et $\mathcal{F}$ sont dit \textit{isomorphes} s'il existe une application affine bijective entre eux.
\end{mydef}

\begin{myproposition}
	Soit $\mathcal{E}$ un espace affine, il est isomorphe à $\overrightarrow{\mathcal{E}}$.
\end{myproposition}

En effet, en fixant $O \in \mathcal{E}$, on définit $\funcshort{f}{\overrightarrow{\mathcal{E}}}{\mathcal{E}}$ définie par $f(u) := O + u$ pour tout $u \in \overrightarrow{\mathcal{E}}$

\begin{mydef}
	Une suite exacte est une suite de la forme
	$$G_1 \xrightarrow{f_1} G_2 \xrightarrow{f_2} ... G_n$$
	où pour tout $i$, $\ker f_{i+1} = Im (f_i)$
\end{mydef}

\begin{myexmpl}
	\begin{itemize}
		
		\item Si $\{e\} \rightarrow G \xrightarrow{f} H$ est exacte, alors $f$ est injective.
	
		\item Si $G \rightarrow H \xrightarrow{f} \{e\}$ est exacte, alors $f$ est surjective.

		\item $\{\text{groupe des translations}\} \rightarrow GA(\mathcal{E}) \rightarrow_\varphi GL(E)$ où $\varphi$ associe à une application affine son application linéaire associée.
	\end{itemize}
\end{myexmpl}

\begin{mydef}
	Soit $\mathcal{E}$ un espace affine de dimension $E$, $(\lambda_i)_{i \leqslant n}$ des scalaires de somme égale à 1 et $(A_i)_{i \leqslant n}$, on appelle \textit{barycentre} des points pondérés $((A_i, \lambda_i))_{i \leqslant n}$ l'unique point $G$ tel que
	$$\sum_{i = 1}^{n} \lambda_i \overrightarrow{GA_i} = 0$$
	Plus précisément pour tout $O \in \mathcal{E}$
	$$\sum \lambda_i \overrightarrow{OA_i} = \overrightarrow{OG}$$
	Par convention on note
	$$\sum \lambda_i A_i$$
\end{mydef}

\begin{mythm}Soit $\mathcal{F} \subseteq \mathcal{E}$
	\begin{enumerate}
		\item $\mathcal{F}$ est un sous-espace affine s'il est stable par barycentre
		\item $\funcshort{f}{\mathcal{E}}{\mathcal{F}}$ est affine si et seulement si elle préserve les barycentres
	\end{enumerate}
\end{mythm}

\begin{myproperty}
	Soit $(A_{i, j})$ une famille de suite de points. de coefficients $(\lambda_{i, j})$, $(G_i)$ leurs barycentres et $G$ le barycentre des barycentres affectés des poids $(\mu_i)$, alors $$G = \sum \lambda_{j, i} \mu_j A_{j, i}$$
\end{myproperty}

\end{document}